\chapter{Einleitung}
\label{cha:Einleitung}

Wir leben in der Zeit Künstlicher Intelligenz, Big Data und intelligenter Robotersysteme. Heutzutage ist die Vielfalt der Robotersysteme sehr beeindruckend: Roboter finden Einsatz in Form autonomer Autos, intelligenter Produktionstechnik und auch \textit{Drohnen}. Vor 5 Jahren waren Drohnen noch manuell gesteuert, und erlaubten keine anspruchsvolle Rechenoperationen direkt auf der Hardware. Zukünftig sollen Drohnen leistungsfähiger sein und noch autonomer funktionieren und das nicht nur einzeln sondern auch in Schwärmen. Der Fokus dieser Arbeit liegt im Gebiet der Drohnenschwärme, genauer gesagt in der Betrachtung dieser als \textit{verteiltes System}. Dadurch, dass Drohnen miteinander kommunizieren, können sie in einem Schwarm voll autonom und ohne Kollisionen fliegen. Die Kommunikation im Schwarm liefert in diesem Fall zusätzliche Informationen für die Funktion des autonomen Fliegens. Die Anwendung von Algorithmen für verteilte Systeme macht diese Bachelorarbeit auch dadurch interessant, dass diese Algorithmen mehr Anwendungsfälle für Interaktionen innerhalb eines Schwarmes und auch zwischen einem Schwarm und der Cloud bieten. Ein weiterer möglicher Anwendungsfall ist die Entlastung von Operatoren, die diese Drohnen normalerweise steuern. Die Entwicklung von Drohnen mit einem hohen Grad an Autonomie bietet den Operatoren im Laufe einer Mission bzw. eines Auftrages mehr Freiheit, andere nützliche Aufgaben zu erfüllen (Kamera ausrichten, Objekt beobachten, 3D Mapping des Geländes steuern, usw).

Fortschritte in Kommunikationstechnologien (vor allem 5G), Künstlicher Intelligenz und eingebetteter Hardware haben die zweite Welle von \textit{Drohnentechnologie} ausgelöst. Manche Anwendungsfälle, die früher nicht denkbar waren, werden derzeit in Forschungsinstituten und Universitäten (in Europa vor allem EPFL  \footnote{Die École polytechnique fédérale de Lausanne ist eine technisch-naturwissenschaftliche Universität in Lausanne, Schweiz.} und ETH Zürich) erforscht. Innovative Startups versuchen, all diesen Anwendungsfälle auf den Markt zu bringen. Die Drohnenindustrie bietet bahnbrechende Anwendungsfälle und dadurch auch neue Herausforderungen.

Das Problem, das diese Arbeit löst befasst sich mit einer solchen Herausforderung, und zwar \textit{der Synchronisation zwischen mehreren Knoten in einem Cluster von Drohnen}. Wenn Drohnen eine abstrakte Kommunikationstechnologie verwenden, muss ein Algorithmus gewählt werden, der Daten von Drohnen in einem Schwarm zuverlässig synchronisiert. Um eine passende Lösung dafür zu finden, wurden bestimmte Algorithmen verglichen. Zwei Algorithmen, die sich bei der Betrachtung verteilter Systeme und der Synchronisation von Konten besonders auszeichnen, sind \textit{Paxos} und \textit{Raft}. Das Problem der Synchronisation zwischen den Knoten eines Systems wird in der Informatik auch \textit{"Problematik der Replikation des Zustandsautomaten"} genannt. Langjährige Erfahrungen der Cloudprovider, die diese Algorithmen für mehrere Bereiche ihrer Infrastruktur einsetzen (vor allem Speicherdienste und \textit{MapReduce} \footnote{MapReduce ist ein Programmiermodell für nebenläufige Berechnungen über große Datenmengen auf Computerclustern.} Clusters), haben die Nutzbarkeit dieser Algorithmen bewiesen. Zwar werden beide Algorithmen in zahlreichen Produktivsystemen benötigt, die Meinungen über die jeweiligen Stärken und Schwächen der Algorithmen sind jedoch noch gespalten.

In einem engen Zusammenhang mit verteilten Systemen stehen \textit{Concurrency Modelle}. Ergänzend zu den unabhängigen Fehlern der Teilkomponenten und dem Fehlen der Global Clock ist die Concurrency eine der wichtigsten Eigenschaften der verteilten Systeme. In dieser Bachelorarbeit werden die Eigenschaften mehrerer Concurrency Modelle verglichen, um ein dafür am besten geeignetes Modell für die Entwicklung verteilter Systeme zu finden.

Überblick über das Dokument: im 2. Kapitel~\ref{cha:Analyse} wird das Thema der Drohnen und der Interaktion von Drohnen in einem Schwarm betrachtet: was sind die Vor- und Nachteile bei der Verwendung eines Schwarms, welche Schwierigkeiten bringt die Steuerung eines Schwarms mit sich, welche theoretischen Frameworks gibt es für die Steuerung verteilter Agenten. Außerdem wird das Thema der verteilten Systeme behzandelt: theoretische Grundlagen werden erläutert und Algorithmen für verteilte Systeme verglichen. Anschließend werden Concurrency Modelle für die Implementierung verteilter und ausfallsicherer Systeme untersucht. Das letzte Kapitel der Arbeit~\ref{cha:Entwicklung} befasst sich mit der Umsetzung eines der besprochenen Algorithmen mit einem der ausgewählten Concurrency Modelle. In diesem Kapitel wird der ausgewählte Algorithmus im Detail analysiert und erklärt. Welche Entscheidungen wurden bei der Umsetzung getroffen und weshalb. Zudem zeigt dieses Kapitel, welche Vorteile die ausgewählten Werkzeuge bieten.
