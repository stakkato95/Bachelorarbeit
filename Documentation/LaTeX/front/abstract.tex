\chapter{Abstract}


\begin{english} %switch to English language rules

Drones change everyday life in surprising ways: they save in disaster areas, provide medicine in conflict zones, reduce traffic flow and emissions, and make risky jobs safer. Today, drones are used in many ways. Other areas where drones are used are the army (control of swarms of drones), the telecommunications sector (on demand network provisioning) and training (360-degree videos).

There are several research areas that focus on optimizing individual drones operation and swarms of drones control. One of these research topics is swarming. Swarming is a subfield of multi-agent systems that allows drones to mimic behavior of certain animal species. Swarm intelligence allows hundreds, if not thousands, of drones to work together to complete challenging tasks. Today the entire robotics industry is doing research in the field of collaborative robotics, which enables robots to work side by side with humans and to be trained by humans. In the near future, however, we will be living in the age of cloud robotics, in which robots will act and think as a single organism without human interaction. Advances in artificial intelligence and cloud robotics are driving swarm technology and will help drones communicate not only with their operators, but also with each other. Nowadays, each drone is controlled by one or more people. Tomorrow drones may not need operators at all. Researchers are investigating possibilities of replacing humans with artificial intelligence and algorithms for distributed systems. As embedded systems become more powerful each year, they get more and more capacity to process sensor data without cloud connection. Innovative approaches to building swarm systems suggest splitting computation evenly across all drones and redundantly securing drones which fulfill special functions in a swarm or carry unique hardware.

All of these functions require classic algorithms for distributed systems that were used in cloud solutions for years. They are based on the algorithms from the 80s and 90s, which are still used in production systems and serve as inspiration for more optimal algorithms. The goal of this bachelor thesis is to investigate available algorithms for distributed systems and find an optimal concurrency model for robust and fault-tolerant embedded drone systems.


\end{english}

