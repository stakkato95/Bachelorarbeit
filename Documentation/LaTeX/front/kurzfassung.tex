\chapter{Kurzfassung}


Drohnen ändern den Alltag auf überraschende Weise: sie retten in Katastrophengebieten, sichern die medizinische Versorgung in Konfliktzonen, reduzieren Verkehr und Emissionen und machen riskante Berufe sicherer. Heutzutage werden Drohnen vielseitig eingesetzt. Andere Gebiete, bei denen Drohnen zum Einsatz kommen, sind das Militär (Steuerung von Drohnenschwärmen), der Telekommunikationsbereich (schneller Aufbau von Netzwerken auf Nachfrage) und die Ausbildung (Aufnahme von 360-Grad-Videos).


\noindent
Es gibt mehrere Forschungsrichtungen, die Funktionsweise einzelner Drohnen und Drohnenschwärmen optimieren. Eine von dennen ist Swarming. Swarming ist ein Bereich von Multiagentensystem, der Drohnen ermöglicht, bestimmte Tierarten nachzuahmen. Die Schwarmintelligenz ermöglicht es hunderten, wenn nicht tausenden von Drohnen, zusammenzuarbeiten, um herausfordernde Aufgaben zu erledigen. Heutzutage forscht die gesamte Robotikindustrie auf dem Gebiet der kollaborativen Robotik, bei der Roboter nebeneinander arbeiten und von Menschen trainiert werden. Schon in Naher Zukunft werden wir aber in der Zeit der Cloudrobotik leben, in der Roboter ohne menschlicher Interaktion als ein einziger Organismus agieren und denken werden. Fortschritte in den Bereichen der Künstlicher Intelligenz und der Cloudrobotik treiben die Schwarmtechnologie an und werden dazu beitragen, dass Drohnen nicht nur mit ihren Operatoren, sondern auch miteinander kommunizieren werden. Heutzutage wird jede Drohne von einem oder mehreren Menschen gesteuert. Die Drohnen von morgen werden möglicherweise überhaupt keine Operatoren brauchen. Forscher untersuchen Möglichkeiten, Menschen durch Künstliche Intelligenz und Algorithmen für verteilte Systeme zu ersetzen. Da eingebettete Systeme mit jedem Jahr leistungsstärker werden, haben sie genug Kapazitäten, um Sensordaten ohne Cloudverbindung zu bearbeiten, Rechenoperationen gleichmäßig untereinander aufzuteilen und Drohnen mit speziellen Funktionen im Schwarm redundant abzusichern. 


\noindent
Alle diesen Funktionen brauchen klassische Algorithmen für verteilte Systeme, die schon seit Jahren bei Cloudlösungen zum Einsatz kommen. Sie beruhen auf den Algorithmen aus den 80ern und 90ern, die immer noch in Produktivsystemen verwendet werden und als Inspiration für neue und optimalere Algorithmen dienen. Der Fokus dieser Bachelorarbeit liegt auf der Untersuchung verfügbarer Algorithmen für verteilte Systeme und auf der Recherche nach einem optimalen Concurrency Modell für robuste und fehlertolerante eingebettete Systeme.
